\begin{enumerate}[label=\thesubsection.\arabic*,ref=\thesubsection.\theenumi]
    \item The sum of series $\frac{1}{2!}-\frac{1}{3!}+\frac{1}{4!}-\dots$ upto infinity is 
    \hfill(2007)
%
    \begin{enumerate}
    \item$e^{-\frac{1}{2}}$
    \item$e^{+\frac{1}{2}}$
    \item$e^{-2}$
    \item$e^{-1}$
    \end{enumerate}
\item For a positive integer $n$,  let
	$a\brak{n}=1+\frac{1}{2}+\frac{1}{3}+\frac{1}{4}+\dots+\frac{1}{2^n-1}$. Then \hfill\brak{1999}
\begin{multicols}{4}
\begin{enumerate}    
\item $a(100)\leq 100$
\item $a(100) > 100$
\item $a(200)\leq 100$
\item $a(200) > 100$
\end{enumerate}
\end{multicols}
%
%
\item Let 
\begin{align*}
S_n=\sum_{k=1}^{n}\frac{n}{n^2+kn+k^2} \text{ and }   T_n=\sum_{k=0}^{n-1}\frac{n}{n^2+kn+k^2}
\end{align*}
for $n=1, 2, 3, \dots$ Then, \hfill\brak{2008}
\begin{multicols}{4}
\begin{enumerate}    
\item $S_n<\frac{\pi}{3\sqrt{3}}$
\item $S_n>\frac{\pi}{3\sqrt{3}}$
\item $T_n<\frac{\pi}{3\sqrt{3}}$
\item $T_n>\frac{\pi}{3\sqrt{3}}$
\end{enumerate}
\end{multicols}
%
\item {The sum of the series \\ $\frac{1}{1\cdot2}-\frac{1}{2\cdot3}+\frac{1}{3\cdot4}\cdots \text{ up to } \infty$ is equal to} 
{\hfill{\sbrak{2003}}} 
\begin{enumerate}
\begin{multicols}{2}
\item  {$\log_e \brak{\frac{4}{e}}$}
\item  {2$\log_e 2$}
\item  {$\log_e 2-1$}
\item  {$\log_e 2$}
\end{multicols}
\end{enumerate}

\item {The sum of series $\frac{1}{2!}+\frac{1}{4!}+\frac{1}{6!}+\cdots$ is}
{\hfill{\sbrak{2004}}} 
\begin{enumerate}
\begin{multicols}{2}

\item  {$\frac{\brak{e^2-2}}{e}$}
\item  {$\frac{n^2\brak{n+1}}{2}$}
\item  {$\frac{n\brak{n+1}^2}{2e}$}
\item  {$\frac{\brak{e^2-1}}{2}$}
\end{multicols}
\end{enumerate}

\item {The sum of the series $1+\frac{1}{4\cdot2!}+\frac{1}{16\cdot4!}+\frac{1}{64\cdot6!}+\cdots  \infty$ is}
{\hfill{\sbrak{2005}}} 
\begin{enumerate}
\begin{multicols}{2}
\item  {$\frac{e-1}{\sqrt{e}}$}
\item  {$\frac{e+1}{\sqrt{e}}$}
\item  {$\frac{e-1}{2\sqrt{e}}$}
\item  {$\frac{e+1}{2\sqrt{e}}$}
\end{multicols}
\end{enumerate}

    \item The sum of series $\frac{1}{2!}-\frac{1}{3!}+\frac{1}{4!}-\dots$ upto infinity is 
    \hfill(2007)

    \begin{enumerate}
    \item$e^{-\frac{1}{2}}$
    \item$e^{+\frac{1}{2}}$
    \item$e^{-2}$
    \item$e^{-1}$
    \end{enumerate}
    \item For any positive integer $n$, let $S_n: (0, \infty) \to \mathbb{R}$ be defined by  
    \[
    S_n(x) = \sum_{k=1}^n \cot^{-1}\left(\frac{1 + k(k+1)x^2}{x}\right),
    \]  
    where for any $x \in \mathbb{R}$, $\cot^{-1}(x) \in (0, \pi)$ and $\tan^{-1}(x) \in \left(-\frac{\pi}{2}, \frac{\pi}{2}\right)$. Then which of the following statements is (are) TRUE?  
    \hfill(2021)
    \begin{enumerate}
        \item  $S_{10}(x) = \frac{\pi}{2} - \tan^{-1}\left(\frac{1 + 11x^2}{10x}\right)$, for all $x > 0$  
        \item  $\lim_{n \to \infty} \cot(S_n(x)) = x$, for all $x > 0$  
        \item  The equation $S_3(x) = \frac{\pi}{4}$ has a root in $(0, \infty)$  
        \item  $\tan(S_n(x)) \leq \frac{1}{2}$, for all $n \geq 1$ and $x > 0$
    \end{enumerate}
\end{enumerate}
